\documentclass[tikz, border=1cm]{standalone}
%%%%%%%%%%%%%%
\usepackage{tikz}
\usepackage{pgfplots}
\usepgfplotslibrary{fillbetween}
%%%%%%%%%%%%%%

%%%%%%%%%%%%%%
\definecolor{blue1}{RGB}{0,177,234}
\definecolor{blue4}{RGB}{178,231,248}
\definecolor{gray1}{RGB}{76,84,93}
\definecolor{gray4}{RGB}{201,203,206}
\definecolor{bluegray1}{RGB}{0,127,167}
\definecolor{bluegray4}{RGB}{178,216,228}
\definecolor{orange1}{RGB}{255,126,46}
\definecolor{orange4}{RGB}{255,216,192}
\definecolor{purple1}{RGB}{89,89,171}
\definecolor{purple4}{RGB}{189,189,231}
%%%%%%%%%%%%%%

\pgfplotsset{compat=1.17}

%%% Patiente
\def\name{Dupont}

%%% Enter LF values below as a succession of (week, lf) pairs
%%% Example: \def\lfi{(15,28) (20,47) (25,62) (30,73) (35,82)}
%%% Leave blank to skip bip plot
%%% For twins, fill the lfii line in the same way
\def\lfi{(15,18) (20,31) (25,44) (30,58) (35,65)}
\def\lfii{(15,16) (20,28) (25,39) (30,52) (35,62)}

%%% Enter BIP values below as a succession of (week, bip) pairs
%%% Example: \def\bipi{(15,28) (20,47) (25,62) (30,73) (35,82)}
%%% Leave blank to skip lf plot
%%% For twins, fill the bipii line in the same way
\def\bipi{(15,28) (20,47) (25,62) (30,73) (35,82)}
\def\bipii{}

%%% Enter PA values below as a succession of (week, pa) pairs
%%% Example: \def\pai{(15,28) (20,47) (25,62) (30,73) (35,82)}
%%% Leave blank to skip pa plot
%%% For twins, fill the paii line in the same way
\def\pai{(15,100) (20,150) (25,205) (30,255) (35,306)}
\def\paii{}

%%% Enter PC values below as a succession of (week, pc) pairs
%%% Example: \def\pci{(15,28) (20,47) (25,62) (30,73) (35,82)}
%%% Leave blank to skip pa plot
%%% For twins, fill the pcii line in the same way
\def\pci{(20,170) (25,230) (30,275) (35,305)}
\def\pcii{}

%%%%%%%%%%%%%%
\begin{document}
%%%%%%%%%%%%%%
\begin{tikzpicture}[	trim axis left, trim axis right, font=\scriptsize,
				upper/.style=	{name path=upper, smooth},
				lower/.style=	{name path=lower, smooth},
				p3p97/.style=	{smooth, dash pattern=on 2pt},
				p10p90/.style=	{opacity=0.5},
				p50/.style=	{thick, smooth}]
	\begin{axis}[	xmin=11, xmax=41, ymin=0, ymax=160,
				legend cell align=left, legend pos=south west,
				ytick={0,10,20,30,40,50,60,70,80,90,100,110,120,130,140,150,160},
				grid=both, minor tick num=4, 
				major grid style={line width=.5pt,draw=gray!50}, 
				minor grid style={line width=.1pt,draw=gray!50},
				width=10cm, height=15cm, title=Mme \name]
		
		% bip
		\addplot[p3p97, 	forget plot, draw=orange1] 	table[x index=0,y index=1] {bip.csv}; 
		\addplot[p3p97, 	forget plot, draw=orange1] 	table[x index=0,y index=5] {bip.csv};
		\addplot[upper, 		forget plot, draw=orange1] 	table[x index=0,y index=2] {bip.csv}; 
		\addplot[lower, 		forget plot, draw=orange1] 	table[x index=0,y index=4] {bip.csv}; 
		\addplot[p10p90, 	forget plot, fill=orange4] 		fill between[of=upper and lower];
		\addplot[thick, smooth,		 draw=orange1]		table[x index=0,y index=3] {bip.csv}; \label{bip}
		\addplot [mark=*,mark size=1.5pt, forget plot] coordinates {\bipi};
		\addplot [mark=o,mark size=1.5pt, forget plot] coordinates {\bipii};
		
		% lf
		\addplot[p3p97, 	forget plot, draw=blue1] 	table[x index=0,y index=1] {lf.csv}; 
		\addplot[p3p97, 	forget plot, draw=blue1] 	table[x index=0,y index=5] {lf.csv};
		\addplot[upper, 		forget plot, draw=blue1] 	table[x index=0,y index=2] {lf.csv}; 
		\addplot[lower, 		forget plot, draw=blue1] 	table[x index=0,y index=4] {lf.csv}; 
		\addplot[p10p90, 	forget plot, fill=blue4] 	fill between[of=upper and lower];
		\addplot[thick, smooth,		 draw=blue1]	table[x index=0,y index=3] {lf.csv}; \label{lf}
		\addplot [mark=*,mark size=1.5pt, forget plot] coordinates {\lfi};
		\addplot [mark=o,mark size=1.5pt, forget plot] coordinates {\lfii};
		
	\end{axis}
	
	\begin{axis}[	xmin=11, xmax=41, ymin=0, ymax=400, axis y line*=right,
				legend cell align=left, legend pos=north west,
				ytick={0,25,50,75,100,125,150,175,200,225,250,275,300,325,350,375,400},
				grid=both, minor tick num=4,
				major grid style={line width=.5pt,draw=gray!50}, 
				minor grid style={line width=.1pt,draw=gray!50},
				width=10cm, height=15cm]

		\addlegendimage{/pgfplots/refstyle=lf}	\addlegendentry{\textsc{lf}}
		\addlegendimage{/pgfplots/refstyle=bip}	\addlegendentry{\textsc{bip}}
		
		% pa
		\addplot[p3p97, 	forget plot, draw=purple1] 	table[x index=0,y index=1] {pa.csv}; 
		\addplot[p3p97, 	forget plot, draw=purple1] 	table[x index=0,y index=5] {pa.csv};
		\addplot[upper, 		forget plot, draw=purple1] 	table[x index=0,y index=2] {pa.csv}; 
		\addplot[lower, 		forget plot, draw=purple1] 	table[x index=0,y index=4] {pa.csv}; 
		\addplot[p10p90, 	forget plot, fill=purple4] 	fill between[of=upper and lower];
		\addplot[thick, smooth,		 draw=purple1]	table[x index=0,y index=3] {pa.csv}; \addlegendentry{\textsc{pa}}
		\addplot [mark=*,mark size=1.5pt, forget plot] coordinates {\pai};
		\addplot [mark=o,mark size=1.5pt, forget plot] coordinates {\paii};
		
	\end{axis}	
	
\end{tikzpicture}%
%%%%%%%%%%%%%%
%%%%%%%%%%%%%%
\begin{tikzpicture}[	trim axis left, trim axis right, font=\scriptsize,
				upper/.style=	{name path=upper, smooth},
				lower/.style=	{name path=lower, smooth},
				p3p97/.style=	{smooth, dash pattern=on 2pt},
				p10p90/.style=	{opacity=0.5},
				p50/.style=	{thick, smooth}]
	\begin{axis}[	xmin=11, xmax=41, ymin=0, ymax=160,
				legend cell align=left, legend pos=south west,
				ytick={0,10,20,30,40,50,60,70,80,90,100,110,120,130,140,150,160},
				grid=both, minor tick num=4, 
				major grid style={line width=.5pt,draw=gray!50}, 
				minor grid style={line width=.1pt,draw=gray!50},
				width=10cm, height=15cm, title=Mme \name]
		
		% bip
		\addplot[p3p97, 	forget plot, draw=orange1] 	table[x index=0,y index=1] {bip.csv}; 
		\addplot[p3p97, 	forget plot, draw=orange1] 	table[x index=0,y index=5] {bip.csv};
		\addplot[upper, 		forget plot, draw=orange1] 	table[x index=0,y index=2] {bip.csv}; 
		\addplot[lower, 		forget plot, draw=orange1] 	table[x index=0,y index=4] {bip.csv}; 
		\addplot[p10p90, 	forget plot, fill=orange4] 		fill between[of=upper and lower];
		\addplot[thick, smooth,		 draw=orange1]		table[x index=0,y index=3] {bip.csv}; \label{bip}
		\addplot [mark=*,mark size=1.5pt, forget plot] coordinates {\bipi};
		\addplot [mark=o,mark size=1.5pt, forget plot] coordinates {\bipii};
		
	\end{axis}
	
	\begin{axis}[	xmin=11, xmax=41, ymin=0, ymax=400, axis y line*=right,
				legend cell align=left, legend pos=north west,
				ytick={0,25,50,75,100,125,150,175,200,225,250,275,300,325,350,375,400},
				grid=both, minor tick num=4,
				major grid style={line width=.5pt,draw=gray!50}, 
				minor grid style={line width=.1pt,draw=gray!50},
				width=10cm, height=15cm]

		\addlegendimage{/pgfplots/refstyle=bip}	\addlegendentry{\textsc{bip}}
		
		% pc
		\addplot[p3p97, 	forget plot, draw=purple1] 	table[x index=0,y index=1] {pc.csv}; 
		\addplot[p3p97, 	forget plot, draw=purple1] 	table[x index=0,y index=5] {pc.csv};
		\addplot[upper, 		forget plot, draw=purple1] 	table[x index=0,y index=2] {pc.csv}; 
		\addplot[lower, 		forget plot, draw=purple1] 	table[x index=0,y index=4] {pc.csv}; 
		\addplot[p10p90, 	forget plot, fill=purple4] 	fill between[of=upper and lower];
		\addplot[thick, smooth,		 draw=purple1]	table[x index=0,y index=3] {pc.csv}; \addlegendentry{\textsc{pc}}
		\addplot [mark=*,mark size=1.5pt, forget plot] coordinates {\pci};
		\addplot [mark=o,mark size=1.5pt, forget plot] coordinates {\pcii};
		
	\end{axis}	
	
\end{tikzpicture}
%%%%%%%%%%%%%%
\end{document}