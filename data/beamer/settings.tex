%%%%%%%%%%%%%%%%%%%%%%%% 
% Main packages
%%%%%%%%%%%%%%%%%%%%%%%%
\usepackage[british]{babel}							% manages multilingual typesetting
\usetheme{metropolis}								% use custom style
\setbeamersize{text margin left=0.5cm}					% left margin 
\setbeamersize{text margin right=0.5cm}					% right margin
\setbeamercovered{transparent}						% style for cover/uncover
\usepackage{etex} 									% extends memory

%%%%%%%%%%%%%%%%%%%%%%%% 
% Fonts
%%%%%%%%%%%%%%%%%%%%%%%% 
\usepackage[defaultfam,extralight,tabular,lining]{montserrat}
\usepackage[T1]{fontenc}
\renewcommand*\oldstylenums[1]{{\fontfamily{Montserrat-TOsF}\selectfont #1}}

%%%%%%%%%%%%%%%%%%%%%%%% 
% Math packages
%%%%%%%%%%%%%%%%%%%%%%%% 
\usepackage{amsmath}								% for main mathematical features
\usepackage{amsxtra}								% a few rarely used math-related commands
\usepackage{amssymb}								% extended symbol collection
\usepackage{amsthm}								% for theorem-like structures
\usepackage{ifthen}									% conditional commands
\usepackage{mathtools}								% extension of amsmath
\usepackage{stmaryrd} 								% symbols for computer science
\usepackage{siunitx} 								% formatting numbers with 10^
\usepackage{textcomp} 								% for many text symbols
\usepackage{bm} 									% boldmath symbols
\usepackage{esint} 									% for \oiint symbol and integrals style
\usepackage{fp}									% complex calculations
\usepackage[beamer,customcolors]{hf-tikz}				% highlight formulas
\usefonttheme[onlymath]{serif}							% use serif math font in beamer

%%%%%%%%%%%%%%%%%%%%%%%% 
% Arrays packages
%%%%%%%%%%%%%%%%%%%%%%%% 
\usepackage{booktabs}								% enhances tables quality
\usepackage{multirow}								% create tab­u­lar cells span­ning mul­ti­ple rows
\usepackage{multicol}								% create tab­u­lar cells span­ning mul­ti­ple columns
\usepackage{bigdelim}								% multi-row delimiters in arrays
\usepackage{tabularx}								% extends tabular with adjustable-width columns 

%%%%%%%%%%%%%%%%%%%%%%%% 
% Layout packages
%%%%%%%%%%%%%%%%%%%%%%%% 
\usepackage{hyperref}								% production of hypertext links
\usepackage{enumitem}								% provides control over itemize and enumerate packages
\usepackage{ulem} 									% for underlining things
\usepackage{algorithm}								% typesetting for algorithms
\usepackage{algpseudocode}							% provides control over algorithm-defined structures
\usepackage{setspace}								% to set space between lines
\usepackage{lipsum}									% write random text
\usepackage{listings}								% to typeset source code
\usepackage{bibentry}								% allows to put a full bib entry in the text of a document
\usepackage{pdflscape}								% to use landscape mode
\usepackage{fix-cm}									% drops latex size restrictions over Modern fonts
\usepackage{appendixnumberbeamer}					% manage frame numbering in beamer appendices
\usepackage{anyfontsize}								% allows to use any font size

%%%%%%%%%%%%%%%%%%%%%%%% 
% Figures packages
%%%%%%%%%%%%%%%%%%%%%%%% 
\usepackage{graphicx}								% graphics support
\usepackage[font={footnotesize},labelfont={bf}]{caption} 		% style of figure captions
\usepackage{fp}									% high-precision arithmetic operations
\usepackage[final]{pdfpages} 							% inclusion of PDF pages (must be put BEFORE pgfplots, tikz and externalize stuff)
\usepackage{subcaption}								% for subfigures
\usepackage{pgfplots}								% pgfplots support
\usepackage{pgfplotstable}							% rounding and formatting of numerical tables for pgfplots
\usepackage{tikz}									% tikz support
\usepackage{tikz-3dplot}								% 3D plots
\usepackage{animate}                                 % Videos
\usepgfplotslibrary{fillbetween}							% filling between curves
\usepgfplotslibrary{polar}								% polar plots
\usepgfplotslibrary{patchplots}							% plot triangular or quadrangular patches on graphs
\usetikzlibrary{calc}									% complex coordinates calculations
\usetikzlibrary{fadings}								% to apply fadings
\usetikzlibrary{shapes}								% to define various complex shapes
\usetikzlibrary{shapes.misc}							% to define various complex shapes
\usetikzlibrary{shapes.geometric}						% to define geometric shapes such as polygons or ellipses 
\usetikzlibrary{3d}									% drawing 3d shapes
\usetikzlibrary{arrows}								% arrow tips
\usetikzlibrary{arrows.meta}							% arrow tips
\usetikzlibrary{quotes}								% nice quotes
\usetikzlibrary{decorations}							% basic decoration stuff
\usetikzlibrary{decorations.pathmorphing}					% deformation of to-be-decorated paths
\usetikzlibrary{decorations.text}							% decoration of paths with text
\usetikzlibrary{decorations.markings}						% putting marks on paths
\usetikzlibrary{intersections}							% automated calculations of paths intersections
\usetikzlibrary{patterns}								% defines patterns for filling areas
\usetikzlibrary{spy}									% to create local close-ups on plots
\usetikzlibrary{chains}								% to create chains (timeline for example)
\usetikzlibrary{positioning}								% additional options for placing nodes
\usetikzlibrary{scopes}								% complex scopes in tikz
\pgfplotsset{compat=1.15}								% require a specific version of pgfplots

\tikzset{every mark/.append style={scale=0.6}}				% size of tikz marks
\pgfplotsset{legend style={font=\footnotesize}}				% size of legend text
\pgfplotsset{every tick label/.append style={font=\footnotesize}}	% size of ticks in plots

%%%%%%%%%%%%%%%%%%%%%%%% 
% Define custom 3D rotations for tikz-3dplot
% You must use \begin{tikzpicture}[tdplot_main_coords] to make them effective
% The command is \tdplotsetmaincoords{phi}{theta} %first angle from z+, second angle from x+
%%%%%%%%%%%%%%%%%%%%%%%% 
\newcommand{\applyANiceRotationForCells}{\tdplotsetmaincoords{60}{115}}

%%%%%%%%%%%%%%%%%%%%%%%% 
% Color packages and definitions
%%%%%%%%%%%%%%%%%%%%%%%% 
\usepackage{xcolor}									% driver-independant colors extensions
\usepackage{color} 									% foreground and background color control 

\definecolor{blue1}		{RGB}{0,177,234}				% deep cyan
\definecolor{blue2}		{RGB}{76,200,239}				% lighter cyan
\definecolor{blue3}		{RGB}{127,215,244}				% lighter cyan
\definecolor{blue4}		{RGB}{178,231,248}				% lighter cyan
\definecolor{blue5}		{RGB}{198,251,255}				% lighter cyan
\definecolor{bluegray1}	{RGB}{0,127,167}				% deep cyan/gray mix
\definecolor{bluegray2}	{RGB}{76,165,193}				% lighter cyan/gray mix
\definecolor{bluegray3}	{RGB}{127,191,211}				% lighter cyan/gray mix
\definecolor{bluegray4}	{RGB}{178,216,228}				% lighter cyan/gray mix
\definecolor{gray1}		{RGB}{76,84,93}				% deep gray
\definecolor{gray2}		{RGB}{129,135,141}				% lighter gray
\definecolor{gray3}		{RGB}{165,169,174}				% lighter gray
\definecolor{gray4}		{RGB}{201,203,206}				% lighter gray
\definecolor{orange1}	{RGB}{255,126,46}				% deep orange
\definecolor{orange2}	{RGB}{255,164,108}				% lighter orange
\definecolor{orange3}	{RGB}{255,190,150}				% lighter orange
\definecolor{orange4}	{RGB}{255,216,192}				% lighter orange
\definecolor{purple}		{RGB}{89,89,171}				% purple
\definecolor{brown}		{RGB}{205,133,63}				% brown

\def\smallopacity	{0.7}								% slightly opaque
\def\mediumopacity	{0.5}								% more opaque
\def\bigopacity		{0.3}								% very opaque

\pgfplotsset{										% custom color bar
    colormap={custom_map}{[5pt]
            rgb255(0pt)=(255,126,46);
            rgb255(500pt)=(255,190,150);
            rgb255(1000pt)=(0,177,234);
            rgb255(1500pt)=(127,215,244);
    },
}

%%%%%%%%%%%%%%%%%%%%%%%% 
% Border of pictures
%%%%%%%%%%%%%%%%%%%%%%%% 
\setlength{\fboxsep}{0pt}
\setlength{\fboxrule}{1pt}

%%%%%%%%%%%%%%%%%%%%%%%% 
% Beamer settings
%%%%%%%%%%%%%%%%%%%%%%%%
\setbeamercolor{titlelike}{parent=palette primary, fg=blue1}
\setbeamercolor{frametitle}{bg=gray2, fg=white}
\setbeamerfont{section in toc}{series=\bfseries}

%%%%%%%%%%%%%%%%%%%%%%%% 
% Custom style for code formatting
%%%%%%%%%%%%%%%%%%%%%%%% 
\lstset{
	language=bash,
	basicstyle=\small\ttfamily\color{bluegray1},
	keywordstyle=\color{blue1},
	commentstyle=\color{purple}
}

%%%%%%%%%%%%%%%%%%%%%%%% 
% Custom default tikz arrow tip
%%%%%%%%%%%%%%%%%%%%%%%%
\tikzset{>=stealth}

%%%%%%%%%%%%%%%%%%%%%%%% 
% Custom enumerate style
%%%%%%%%%%%%%%%%%%%%%%%% 
\setenumerate{label=\textcolor{bluegray1}{$\bm{\diamond}$},itemsep=1pt,topsep=5pt} 

%%%%%%%%%%%%%%%%%%%%%%%% 
% Custom hyperlinks style
%%%%%%%%%%%%%%%%%%%%%%%% 
\hypersetup{
    colorlinks 	= true,								% allow link coloring
    breaklinks 	= true,								% allow linebreaks for long links
    urlcolor 		= purple,								% url color
    linkcolor 	= gray1,								% internal links color
    citecolor 	= bluegray1,							% bibliographic link color
    pdfborder 	= {0 0 0}								% border style
}

%%%%%%%%%%%%%%%%%%%%%%%% 
% Math commands
%%%%%%%%%%%%%%%%%%%%%%%%
\newcommand{\V}[1]{\textbf{#1}}						% vector notation
\newcommand{\GV}[1]{\boldsymbol{#1}}					% vector notation for complex cases

%%%%%%%%%%%%%%%%%%%%%%%% 